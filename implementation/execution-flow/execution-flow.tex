\documentclass{article}
\usepackage{import}
\subimport{../../}{preamble.tex}
\standalonetrue
\begin{document}

% sample content
\subsection{Execution flow}
To get a better overview of how the software itself is structured one can look at the sequence diagram on figure \ref{fig:startasync-sequence-diagram} showing the implementation \mintinline{csharp}{Microsoft.Extensions.Hosting.IHostedService.StartAsync}. The \mintinline{csharp}{Microsoft.Extensions.Hosting.IHostedService} interface is the default way of running background tasks in .NET Core\cite{url:implementation:microsoft:hosted-service}. In this project the spoofers and the services needs to run as a background task as they need to listen for network requests separately from the rest.

\subimport{figures/}{startasync-sequence-diagram.tex}

When starting the background tasks the logic is split into two parts, one for persistent modules and one for action modules. On persistent modules a thread is spun up and the \mintinline{csharp}{Run()} method is executed in that thread. This gives the worker, not the module, control over the thread and can stop it when needed. When persistent modules are started a log message is sent from the \mintinline{csharp}{Worker} to the registered \mintinline{csharp}{IWorkerController} which then first hands it to the datastore for persistent storage and thereafter adds it to the designated hub. Currently three hub exists, UserHub, LogEntryHub and TargetHub.
\\\\
Action modules are special in the way that they need to be run on demand and not as a background task. Therefore the \mintinline{csharp}{Worker} needs a way to tell the \mintinline{csharp}{IWorkerController} which actions exists. This is done using the \mintinline{csharp}{SynchronizeTool()} method which should be implemented in the implementation of \mintinline{csharp}{IWorkerController}.


\end{document}