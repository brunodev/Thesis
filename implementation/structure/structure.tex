\documentclass{article}
\usepackage{import}
\subimport{../../}{preamble.tex}
\standalonetrue
\begin{document}

% sample content
\subsection{Code Structure}
With the considerations from section \ref{sec:considerations} in mind and adhering to the requirements from section \ref{sec:requirements} the structure seen on appendix \ref{sec:appendix:worker-class-diagram} (\nameref{sec:appendix:worker-class-diagram}) and \ref{sec:appendix:web-class-diagram} (\nameref{sec:appendix:web-class-diagram}) is the final structure of the application. Overall the application exists of two distinctive parts, the \mintinline{csharp}{Worker} and the implementation of \mintinline{csharp}{IWorkerController}, \mintinline{csharp}{WebController}. When a \mintinline{csharp}{Worker} is initialized a \mintinline{csharp}{IWorkerController} is passed on as a parameter in the constructor. The project uses the built in version of \gls{di}\cite{url:implementation:microsoft:dependency-injection} for ASP.NET Core to achieve inversion of control, such that the \mintinline{csharp}{Worker} can control its \mintinline{csharp}{IWorkerController}. \gls{di} also adheres to the principles of \gls{idd}.

\subimport{}{worker-structure.tex}
\subimport{}{web-structure.tex}
\subimport{}{vue-frontend-structure.tex}

\end{document}