\documentclass{article}
\usepackage{import}
\subimport{../../}{preamble.tex}
\standalonetrue
\begin{document}

% sample content
\subsubsection{SignalR}
As mentioned in the requirements sections (\ref{sec:requirements}), the web-app needs to use WebSockets for two-way communication between the client and server to ensure that the user does not need to refresh the graphical interface manually. WebSockets is a relatively new technology first seen in the Chrome webbrowser in version 43 which was released in 2015\cite{url:implementation:websockets}. For this projects WebSockets should be used to update the graphical interface in real time with new informations such as log messages, new/changed users and hosts. C\# supports WebSockets natively using the \mintinline{csharp}{Microsoft.AspNetCore.WebSockets} package but external libraries exists to aid developers in using WebSockets. One such library is SignalR\cite{url:implementation:signalr} which is an open-source library that simplifies the creation of real-time web-apps by using WebSockets, Server-Sent events and long polling. SignalR was chosen for its simplicity, popularity and its ecosystem with a lot of active users.

\end{document}