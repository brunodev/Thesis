\documentclass{article}
\usepackage{import}
\subimport{../../}{preamble.tex}
\standalonetrue
\begin{document}

% sample content
\subsubsection{VueJS}
When developing a modern and interactive web UI a JavaScript framework is almost a requirement in itself. A proper framework will make the development process easier and in return create a more stable and fluid frontend. When developing a live web app a frontend framework will also make it easier to ensure that the client and server side is synchronized properly, so that there is no difference in the data possessed by the server and the data shown in the graphical interface.
\\
There exists a wide range of JavaScript libraries with each one having their strengths and weaknesses. The two most popular ones are Angular and React closely followed by Vue\cite{url:implementation:javascript-frameworks}. React and Angular is primarily developed by Facebook and Google respectively, and have a large community surrounding them. Vue on the other hand is the small underdog created by Evan You as an open-source project with the goal of creating a lightweight JavaScript library that could compete with the likes of Angular and React. Vue is often described as the best parts of Angular and React but without all the bloat that comes with such big frameworks. That, and the fact that the author has previously worked with Vue, is the choice for using Vue for this project. Vue works very well for small web-apps with a small developer team, where the code structure is somewhat simple.

%\subimport{}{problem.tex}

\end{document}