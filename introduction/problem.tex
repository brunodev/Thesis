\documentclass{article}
    \usepackage{import}
    \subimport{../}{preamble.tex}
    \standalonetrue
\begin{document}
\subsection{Problem background}
\label{sec:problem-background}
When performing penetration tests on Windows Active Directory domains, one of the goals is usually to obtain Domain Administrator privileges. The first step towards achieving that goal is to gain an initial foothold by obtaining valid domain credentials. Methods such as phishing, password leaks, bruteforce attacks are just some out of many methods ways to achieve this. After obtaining valid credentials, and thereby gaining an initial foothold, many methods exists to escalate your privileges in the Domain such as dumping cached credentials from hosts or exploiting vulnerabilities.
\\\\
To first gain an initial foothold and then escalate your privileges to Domain Administrator privileges is often achieved by using a variety a tools developed. Using these tools can impose a security risk, as the tools are not written by the penetration tester themselves, and therefore the tools can potentially be backdoored or otherwise contain security vulnerabilities.
\\\\
A goal of penetration tests is more often than not to be silent and remain undetected by systems such as \gls{ids}, \gls{siem} and \gls{av}. Publicly available tools are detected by such systems and can therefore in some cases not be used.
\\\\
A challenge of performing penetration tests is to make sure that every action done is documented thoroughly, as this is needed for the customer report afterwards. In most cases a description of how every goal was achieved is required, and without proper documentation this is an impossible task.


\subsection{Problem brief}
The purpose of this project is to determine whether a proper solution to the problem mentioned in section \ref{sec:problem-background} can be found. The project will focus on how to most effectively achieve an initial foothold and then analyze how to escalate the privileges from there. The product will be an application where the necessary functionality is implemented, and the application should contain an easy to user interface.
\\\\
The application should strive to not generate alerts in any \gls{ids}, \gls{siem} and \gls{av} systems, such that the penetration test can achieve the goal of being silent.
\\\\
To accommodate the high level of documentation needed, every action done by the application should be logged and contain full timestamps to achieve full traceability throughout the use of the application. This documentation should be in an easy to read format, such that a timeline can be generated for reporting purposes.
\end{document}