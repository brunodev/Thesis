\documentclass{article}
    \usepackage{import}
    \subimport{../}{preamble.tex}
    \standalonetrue
\begin{document}
\subsection{Penetration testing of Windows Domains}
To fully understand the problem and scope of this project, a clear definition of the term \enquote{penetration test} needs to be established. The meaning of the term penetration test can differ a lot depending on the scope of the assignment. A penetration test can both be scoped as a full blown Red Team exercise where the attackers has to infiltrate the network using vulnerabilities, phishing campaigns or other methods, a simple vulnerability scan with confirmation or something completely different.
\\\\
In this project the term is defined as an internal penetration test, where a computer is placed on the network. This is done to simulate an attack where a rogue device has been plugged into the network or an employee machine has been compromised, but also to simulate an attack where the attacker does not yet have valid credentials to authenticate to the Windows Domain.

\end{document}