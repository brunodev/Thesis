\documentclass{article}
\usepackage{import}
\subimport{../../}{preamble.tex}
\standalonetrue
\begin{document}

% sample content
\subsubsection{\glsentryfull{llmnr}}
\label{sec:llmnr}
\gls{llmnr} is the newest protocol for name resolution where DNS name resolution is not possible\cite{url:rfc:llmnr}. \gls{llmnr} works in the same way as \gls{nbns} in such that a name query is sent to the link-scope multicast address(es), and a responder can hereafter respond to the packet and claim itself as the host that was requested. The sequence of events with \gls{llmnr} according to RFC 4795\cite{url:rfc:llmnr} is the following
\begin{enumerate}
	\item An \gls{llmnr} sender sends a \gls{llmnr} query to the link-scope multicast address(es) on port 5355. This is a \gls{llmnr} packet containing a Question Section
	\item A responder responds to this query by sending an UDP packet. This is a \gls{llmnr} packet containing a Question Section and a Resource Record
	\item The sender process the responders packet
\end{enumerate}

This is all well if we assume that the network itself is not already compromised. In case a malicious host is existent on the network, and the host is listening on the link-scope multicast address there is nothing from stopping this host in responding maliciously to the packets. \gls{llmnr} is made to follow the DNS specification, so in order to spoof it we need to know how DNS packets look like. This can be found in RFC 1035\cite{url:rfc:dns}.
% \paragraph{\gls{llmnr} format}
% \\

\paragraph{\glsentryshort{llmnr} packets}

There exists two different \gls{llmnr} packet types. A \textbf{request} and a \textbf{response}. The \textbf{request} contains the \emph{header} and a \emph{question section}. The \textbf{response} contains a \emph{header}, a \emph{question section} and a \emph{resource record}. Format details of these types can be seen in figure \ref{fig:llmnr-packet}.

\subimport{figures/}{llmnr-packet-specification.tex}

This report will not explain the packet details in full, but will focus on the parts necessary to spoof a \gls{llmnr} response.\\
To answer a \gls{llmnr} packet we need to create a \emph{Resource Record} to match the \emph{Question section} sent out by a client. \emph{Name, Type and Class} should match the request, \emph{TTL} should be set to an arbitrary time in minutes (for example 30 - $0x0000001e$ in bytes), \emph{RDLENGTH} should be 4 ($0x0004$) and \emph{RDATA} should be our own IP Address.

\begin{figure}[H]
	\scriptsize
	\par
	\centering
	\varwidth{\linewidth}
	\verbatiminput{llmnr-request.txt}
	\endvarwidth
    \par
    
    \caption{\gls{llmnr} request}
    \label{fig:llmnr-request}
\end{figure}

\begin{figure}[H]
	\scriptsize
	\par
	\centering
	\varwidth{\linewidth}
	\verbatiminput{llmnr-response.txt}
	\endvarwidth
    \par
    
    \caption{Spoofed \gls{llmnr} response}
    \label{fig:llmnr-response}
\end{figure}

Figure \ref{fig:llmnr-request} and \ref{fig:llmnr-response} shows a valid request and the corresponding spoofed response for a name resolution for \emph{server1}. After responding to the request with the proper response we would have successfully spoofed the \gls{llmnr} protocol and redirected traffic intended for server1 to our host(The attacker).
%\subimport{}{problem.tex}

\end{document}
