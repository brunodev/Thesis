\documentclass{article}
\usepackage{import}
\subimport{../../}{preamble.tex}
\standalonetrue
\begin{document}

% sample content
\subsubsection{\glsentryfull{llmnr}}
\label{sec:llmnr}
\gls{llmnr} is the newest protocol for name resolution where DNS name resolution is not possible\cite{url:rfc:llmnr}. \gls{llmnr} works in the same way as \gls{nbns} in such that a name query is sent to the link-scope multicast address(es), and a responder can hereafter respond to the packet and claim itself as the host that was requested. The sequence of events with \gls{llmnr} according to RFC 4795\cite{url:rfc:llmnr} is the following
\begin{enumerate}
	\item An \gls{llmnr} sender sends a \gls{llmnr} query to the link-scope multicast address(es). This is a \gls{llmnr} packet containing a Question Section
	\item A responder responds to this query by sending an UDP packet. This is a \gls{llmnr} packet containing a Question Section and a Resource Record
	\item The sender process the responders packet
\end{enumerate}

This is all well if we assume that the network itself is not already compromised. In case a malicious host is existent on the network, and the host is listening on the link-scope multicast address there is nothing from stopping this host in responding maliciously to the packets. \gls{llmnr} is made to follow the DNS specification, so in order to spoof it we need to know how DNS packets look like. This can be found in RFC 1035\cite{url:rfc:dns}.
% \paragraph{\gls{llmnr} format}
% \\

\paragraph{\glsentryshort{llmnr} packets}

There exists two different \gls{llmnr} packet types. A \textbf{request} and a \textbf{response}. The \textbf{request} contains the \emph{header} and a \emph{question section}. The \textbf{response} contains a \emph{header}, a \emph{question section} and a \emph{resource record}. Format details of these types can be seen in figure \ref{fig:llmnr-packet}.

\begin{figure}
	\centering
	\begin{bytefield}[bitwidth=2em]{16}
		\begin{rightwordgroup}{Header}
			\bitheader{0-15}\\
			\bitbox{16}{ID}\\
			\bitbox{1}{QR} & \bitbox{4}{QR} & \bitbox{1}{C} & \bitbox{1}{TC} & \bitbox{1}{T} & \bitbox{1}{Z} & \bitbox{1}{Z}  & \bitbox{1}{Z} & \bitbox{1}{Z}  & \bitbox{4}{RCODE}\\
			\bitbox{16}{QDCOUNT}\\
			\bitbox{16}{ANCOUNT}\\
			\bitbox{16}{NSCOUNT}\\
			\bitbox{16}{ARCOUNT}
		\end{rightwordgroup}
		\\\\
		\begin{rightwordgroup}{Question section}
			\wordbox[lrt]{1}{QNAME} \\
			\wordbox[lrb]{1}{$\cdots$} \\
			\bitbox{16}{QTYPE}\\
			\bitbox{16}{QCLASS}
		\end{rightwordgroup}
		\\\\
		\begin{rightwordgroup}{Resource record}
			\wordbox[lrt]{1}{NAME} \\
			\wordbox[lrb]{1}{$\cdots$} \\
			\bitbox{16}{TYPE}\\
			\bitbox{16}{CLASS} \\
			\wordbox[lrtb]{2}{TTL} \\
			\bitbox{16}{RDLENGTH} \\
			\wordbox[lrt]{1}{RDATA} \\
			\wordbox[lrb]{1}{$\cdots$}
		\end{rightwordgroup}
	\end{bytefield}
	\caption{\glsentryfull{llmnr} packet specification\cite{url:rfc:llmnr}\cite{url:rfc:dns}}
	\label{fig:llmnr-packet}
\end{figure}

This report will not explain the packet details in full, but will focus on the parts necessary to spoof a \gls{llmnr} response. The most important fields of the Question Section and the Resource Record is explained in list \ref{list:llmnr-question-section} and \ref{list:llmnr-resource-record}.
\begin{customlist}
	\begin{description}
		\item[QNAME] A domain name in the following format: A length octet followed by that number of octets
		\item[QTYPE] Two octet code which specify the query type. Usually $0x0001$ for Host address(IP)
		\item[QCLASS] Two octet code which specify the query class. Usually $0x0001$ for Internet (IN)
	\end{description}
	\caption{Question Section}
	\label{list:llmnr-question-section}
\end{customlist}

\begin{customlist}
	\begin{description}
		\item[NAME] See QNAME of \emph{Question Section}
		\item[Type] See TYPE of \emph{Question Section}
		\item[CLASS] See QCLASS of \emph{Question Section}
		\item[TTL] a 32 bit unsigned integer which specify Time To Live in minutes
		\item[RDLENGTH] a 32 bit unsigned integer which specify the number of octets in RDATA
		\item[RDATA] A variable length string of octets. Usually an IP address
	\end{description}
	\caption{Resource record}
	\label{list:llmnr-resource-record}
\end{customlist}

So to answer a \gls{llmnr} packet we need to create a \emph{Resource Record} to match the \emph{Question section} sent out by a client. \emph{Name, Type and Class} should match the request, \emph{TTL} should be set to an arbitrary time in minutes (for example 30 - $0x0000001e$ in bytes), \emph{RDLENGTH} should be 4 ($0x0004$) and RDATA should be our own IP Address.

\begin{figure}[H]
	\scriptsize
	\par
	\centering
	\varwidth{\linewidth}
	\verbatiminput{llmnr-request.txt}
	\endvarwidth
    \par
    
    \caption{\gls{llmnr} request}
    \label{fig:llmnr-request}
\end{figure}

\begin{figure}[H]
	\scriptsize
	\par
	\centering
	\varwidth{\linewidth}
	\verbatiminput{llmnr-response.txt}
	\endvarwidth
    \par
    
    \caption{Spoofed \gls{llmnr} response}
    \label{fig:llmnr-response}
\end{figure}

Figure \ref{fig:llmnr-request} and \ref{fig:llmnr-response} shows how the response should look for a request for \emph{server1}. After sending this response we would have succeeded in spoofing the \gls{llmnr} protocol to send traffic to our host.
%\subimport{}{problem.tex}

\end{document}
