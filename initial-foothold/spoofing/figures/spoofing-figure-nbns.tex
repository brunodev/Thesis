\documentclass{article}
\usepackage{import}
\subimport{../../../}{preamble.tex}
\standalonetrue
\begin{document}
\begin{figure}[H]
	\centering
	\begin{tikzpicture}[
			->, thick,]
		\coordinate (O) at (0,0);

		\node [block] (dnsserver) {DNS Server};
		\node [block] (client) [below = 4cm of dnsserver] {Client};
		\node (multicast) [below right = 0.5cm and 5cm of dnsserver] {Multicast};
		\node [block] (attacker) [right = 7cm of client] {Attacker};

		\draw[-angle 90] ([xshift=-0.4cm]client.north) -- node[midway,sloped,above] {1. Who is \emph{Server1?}} ([xshift=-0.4cm]dnsserver.south) ;
		\draw[-angle 90] ([xshift=0.4cm]dnsserver.south) -- node[midway,sloped,above] {2. Don't know} ([xshift=0.4cm]client.north) ;

        \draw[-angle 90] (client.north east) -- node[midway,sloped,above] {3. Name Query Request (\emph{Server1})} (multicast) ;
        
		\draw[-angle 90] ([yshift=0.4cm] attacker.west) -- node[midway, sloped, above] {4. Name Query Response (\emph{Server1})} ([yshift=0.4cm] client.east);
		\draw[-angle 90] ([yshift=-0.4cm] client.east) -- node[midway, sloped, below] {5. Communication started} ([yshift=-0.4cm] attacker.west);
	\end{tikzpicture}
	\caption{How an attacker spoofs the Name Resolution protocols in order to receive traffic intended for other hosts}
	\label{fig:host-resolution-sequence-and-spoofing}
\end{figure}
\end{document}