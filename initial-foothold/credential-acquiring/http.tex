\documentclass{article}
\usepackage{import}
\subimport{../../}{preamble.tex}
\standalonetrue
\begin{document}

% sample content
\subsubsection{\glsentryfull{http}}
Implementing a NTLM \gls{http} \gls{ssp} is, once again, done by encapsulating the NTLM authentication messages into the \gls{http} protocol. This is done using the headers \textbf{WWW-Authenticate} and \textbf{Authorization} for the server and client respectively. The flow is very similar to that of \gls{smb}, and can be seen on figure \ref{fig:http-ntlm-authentication}. The flow starts with a client accessing a server on the standard \gls{http} port 80\footnote{In theory this can be any arbitrary port but seeing as we're obtaining data based on spoofing we need to listen on port 80}, with the server responding with a 401 Unauthorized and supplying the header \emph{WWW-Authenticate: NTLM} to let the client know that the server supports NTLM authentication. After this the standard NTLM authentication occurs with Negotiate, Challenge and Authenticate messages encapsulated in \gls{http}. One important point to highlight is that the NTLM messages are encoded using base64 as \gls{http} is a text based protocol.
After a successful NTLM Authentication over \gls{http} we are once again left with a Authenticate message containing the necessary information to construct a NetNTLMv2 password hash. This hash can be cracked offline using various techniques.

\subimport{figures/}{http-ntlm.tex}
\paragraph{1. GET Request} The client initiates the connection with a GET request to the server
\paragraph{2. 401 Unauthorized} The server responds the GET requests with a 401 Unauthorized response containing the \gls{http} header \emph{WWW-Authenticate: NTLM} letting the client know that it should authenticate using NTLM
\paragraph{3. NTLM Negotiate} The client responds with another GET request containing the base64 encoded NTLM Negotiate message encapsulated in the Authorization header
\paragraph{4. NTLM Challenge} The server responds with the base64 encoded NTLM Challenge message containing the challenge the client should encrypt in the WWW-Authentication header
\paragraph{5. NTLM Authenticate} The client responds with the base64 encoded NTLM Authenticate message
\paragraph{6. 200 Ok} The server now responds with a 200 Ok which let the client know that the credentials were accepted.

The method has been tested on all major browser (Internet Explorer, Chrome, Firefox and Edge), and works flawlessly in all of them.

\end{document}