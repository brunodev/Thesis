\documentclass{article}
\usepackage{import}
\subimport{../../}{preamble.tex}
\standalonetrue
\begin{document}

% sample content
\subsubsection{\glsentryfull{smb}}
As stated in section \ref{sec:credential-acquiring} we need to implement a NTLM \gls{ssp} in \gls{smb}. In \gls{smb} this is done by encapsulating the NTLM authentication into \gls{smb}.
A \gls{smb} session is first started whereafter NTLM authentication happens and then the \gls{smb} session is continued. For this project we do not need to implement the full \gls{smb} protocol, as we are only interested in the NTLM authentication and the encrypted challenge we get from this message. The full flow of this process can be seen on figure \ref{fig:smb-ntlm-authentication} which starts after a host has successfully been spoofed to redirect traffic to our host.

\subimport{figures/}{smb-ntlm.tex}

The authentication flow of \gls{smb}, which can be seen on figure \ref{fig:smb-ntlm-authentication}, uses the following messages in the mentioned order
\paragraph{1. SMB\_COM\_NEGOTIATE} The client first sends a SMB\_COM\_NEGOTIATE message to negotiate the supported authentication types.
\paragraph{2. SMB\_COM\_RESPONSE} The server responds with a SMB\_COM\_REPONSE stating the used authentication method that both the client and server supported. In our case this will always be set to NTLM authentication as we are not interested in Kerberos authentication\footnote{Kerberos authentication could also be used, as we would get a ticket which can also be bruteforced. The hash algorithm is stronger though, so NTLM is preferred}, or any other authentication for that matter.
\paragraph{3. SMB\_COM\_SESSION\_SETUP\_ANDX request 1} The client sends an encapsulated NTLM Negotiate message to the server.
\paragraph{4. SMB\_COM\_SESSION\_SETUP\_ANDX response 1} The server receives the NTLM negotiate message and replies with a NTLM Challenge
\paragraph{5. SMB\_COM\_SESSION\_SETUP\_ANDX request 2} The client sends a NTLM Authenticate message containing the encrypted challenge
\paragraph{6. Access denied} An access denied response is sent to close the connection to the client. At this point we will have received a NetNTLMv1 or NetNTLMv2 hash, so we don't wish to continue the session.
\\\\
In step \textbf{(5)} we receive an \emph{NTLM Authenticate} message which contains the encrypted challenge. Using the data given in this message, we can construct a NetNTLMv2 password hash which can be cracked offline using various cracking techniques such as bruteforcing or rainbow tables.

\end{document}