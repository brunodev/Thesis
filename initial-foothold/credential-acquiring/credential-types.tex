\documentclass{article}
\usepackage{import}
\subimport{../../}{preamble.tex}
\standalonetrue
\begin{document}

% sample content
\subsubsection{Credential types}
\label{sec:credential-types}
In the windows ecosystem there is a lot on confusion on the different types of hashes and where they are used. This short section aims to describe the different hashes briefly to avoid confusion later on in the project. There exists 4 (or 5 depending on who you ask) different hashes in windows\cite{url:hashes:hash-types}
\paragraph{LM}
Is the original hash type used by Windows dating back to OS/2. LM has a number of shortcomings, but the biggest is that it is a maximum length of 14 characters, which is then split into two 7-character chunks, where each part is then encrypted with the string ``KGS\!\@\#\$\%'' and concatenated. The obvious flaw here is that you only have to crack two 7-character hashes instead of one, which can easily be done with modern GPUs in mere hours.
\paragraph{NT}
NT is the current Windows standard for hashing passwords. This is basically just a MD4 of the little endian UTF-16 of the password. MD4 has its obvious flaws concerning collision attacks, but such attacks and methods are out of scope for this project.
\paragraph{NTLM}
NTLM in a combination of LM and NT hashes in the form \emph{LM:NT}. This hash type can be used in attacks known as pass-the-hash\cite{url:microsoft:pass-the-hash-mitigation} where you can authenticate using the NTLM password instead of the clear-text password. So having the NTLM hash is in most attack scenarios essentially the same as having the clear-text password.
\paragraph{NetNTLMv1}
NetNTLMv1 is Microsofts first attempt at a challenge/response hash between a client and a server. It will use either the NT or LM hash to generate the NetNTLMv1 hash. Once again this uses DES and has obvious flaws allowing you to convert it to three different DES keys which can be cracked with much less computer power and converted into an NTLM hash\cite{url:hashes:netntlmv1-to-ntlm}.
\paragraph{NetNTLMv2}
NetNTLMv2 is the newest challenge/response based hash used for network authentication. This hash uses a 8-byte server and client challenge combined with the current time and domain name to create a more secure hash. It also uses HMAC-MD5 as algorithm, which is more secure than DES.

%\subimport{}{problem.tex}

\end{document}