\documentclass{article}
\usepackage{import}
\subimport{../../}{preamble.tex}
\standalonetrue
\begin{document}

% sample content
\subsection{Credential acquiring}
\label{sec:credential-acquiring}
After successfully spoofing either a \gls{nbns} or \gls{llmnr} request we have now succeeded in imposing as another host, meaning all traffic intended for that host will be directed to us. It is also important to mention that name resolution of non-existing hosts will still be spoofed and therefore be registered by the sender of the request as belonging to us. In Windows, name resolution often happens when trying to request either a \gls{smb} share or a website using \gls{http}. Luckily for us Windows has implemented the \gls{sspi} which allows an application to use various security models available on a computer. The security models works as a \gls{sso} solution that allows users to easily authenticate to various services using their cached credentials\cite{url:microsoft:sspi-model}. We can use this to our advantage by implementing a \gls{ssp} in fake \gls{smb} and \gls{http} services. There exists a couple of different \gls{ssp}'s including Negotiate, NTLM, Kerberos, Digest SSP and others. Microsoft recommends that you use Negotiate as it acts as an application layer between the \gls{sspi} and the different \gls{ssp}'s usually choosing Kerberos over NTLM as it is more secure. Though, we can force our service to use the NTLM \gls{ssp}, which will give us a hash that we can crack offline. The different types of hashes will be discussed in more detail in section \ref{sec:credential-types}. 

\subimport{}{credential-types.tex}
\subimport{}{ntlm.tex}
\subimport{}{smb.tex}
\subimport{}{http.tex}

\end{document}