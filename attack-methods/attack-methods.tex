\documentclass{article}
\usepackage{import}
\subimport{../}{preamble.tex}
\standalonetrue
\begin{document}

% sample content
\section{Attack methods}
After successfully spoofing a victim by abusing \gls{llmnr} and \gls{nbns} and getting a NetNTLMv2 hash by running \gls{smb} and \gls{http} \gls{ssp}'s we are at a point where we need to use more traditional methods to compromise the network further. As mentioned in section \ref{sec:credential-types} both NetNTLMv2 and NetNTLMv1 hashes encrypts the server given challenge with either the NT or LM hash. In other words we can use bruteforcing tools such as hashcat\cite{url:hashes:hashcat-example-hashes} to crack the hashes and gain the original password, which will of course lead to a full user compromise as we then have username and password in clear-text.
\\
In the rare case that a password is sufficiently complex and cannot easily be cracked or otherwise guessed, there still exists methods to utilize spoofing and poisoning to gain unauthorized access to systems. One such method is called NTLM Relaying\cite{url:ntlm-relaying} where you construct your \gls{ssp}'s to relay the credentials to another server/host.\footnote{Relaying credentials is outside the scope of this project, but is something that would work well with the rest of setup}
\\
After we have cracked the password we need to move on to compromising the domain further. There are many was of doing this, and it it very dependent on the targets of the pentest. A common target for many Windows pentests is to gain Domain Admin privileges, either by compromising a user already possessing the privilege or by exploiting various vulnerabilities. This project primary goal is to become Domain Admin in a quiet and efficient way, that does not disturb the network and remain as close to undetectable as possible. Using vulnerabilities is first and foremost a very periodic way to achieving Domain Admin privileges, as most known vulnerabilities are fixed quite easily and we are not in the position of owning zero day vulnerabilities which can be easily burned on standard pentests. Luckily there are many other methods to achieve lateral movement in a Windows network, which this project will explore further in the following sections. Analyzing and discussing all the different types of attack would in itself be a very big project, so in this report we will focus on \gls{lsass} that, among other things, handles storage of credentials.

\subimport{}{lsass.tex}
\subimport{}{remote-access.tex}
\subimport{}{lsass-extraction.tex}

\end{document}