\documentclass{article}
\usepackage{import}
\subimport{../}{preamble.tex}
\standalonetrue
\begin{document}

\subsection{Remote access}
\label{sec:remote-access}
Remote access should be understood in the sense of getting a interactive session on another host that can be used to interact with this host using either a command line or graphical interface. In Windows there are many official and unofficial ways to achieve this. Among official ways you will find methods such as \textbf{Enter-PSSession}, \textbf{Invoke-Command}, \textbf{PSExec}, \textbf{\gls{winrs}}, \textbf{\gls{rdp}} and \textbf{\gls{wmi}}. These all work very well when you have valid credentials with clear-text passwords. But as stated in section \ref{sec:lsass} dumping credentials from \gls{lsass} memory will in newer Windows versions give you a NT hash and not the clear-text password. In these cases we can utilize a technique called \gls{pth}, which will allow us to use NTLM Authentication to authenticate against a remote host using a hashed password.
\paragraph{\glsentryfull{pth}}
\glsentryfull{pth} attacks is a way of authenticating to a remote host using a NT or LM hash instead of of the clear-text password. If we look closer at the \enquote{\nameref{fig:ntlm-authenticate-message}} figure from page \pageref{fig:ntlm-authenticate-message} and the NetNTLMv2 hash from section \ref{sec:credential-types}, we can clearly see that the NetNTLMv2 hash is based on the NT or LM hash and \textbf{not} the clear-text password. Even though Windows default to Kerberos authentication, we can still force the host to accept NTLM Authentication unless it is actively disabled. Though, even with Kerberos authentication we can still use the NT or LM hash to create a ticket which can then be passed, but that is outside the scope of this project and has other implications as it requires direct communication with a \gls{kdc} (Usually a \gls{dc}).
\\
\gls{pth} attacks can be used with most Windows protocols such as \gls{smb} and \gls{wmi}, but it requires you to re-implement the protocol to be able to use it, as Windows did not make the feature available in any of their tools. Luckily the Impacket project has done most of the work already. As it states in their description, \emph{\enquote{Impacket is focused on providing low-level programmatic access to the packets and for some protocols (e.g. SMB1-3 and MSRPC) the protocol implementation itself.}}\cite{url:impacket:github}.
\\
From usage of the Impacket library and the accompanying tools the author has had best result with the WMIExec tool for remote access, and therefore that is the one used in this project.

\paragraph{Impacket WMIExec} WMIExec is a remote access tool from the Impacket suite that supports both clear-text and \gls{pth} authentication. Furthermore it has a semi-interactive shell accompanied, which can be used to upload and download files from the remote host. If we look deeper into the technical process, it works by using \gls{wmi} to execute commands and \gls{smb} to upload/download files. Looking at the source code\cite{url:impacket:wmiexec} we can get an overview of how it works.

\begin{enumerate}
    \item A \gls{smb} connection is established to remote host using NTLM Authentication. The \gls{smb} connection is used every time a file needs to be uploaded or downloaded
    \item If the command parameter is not set, it will do nothing and wait for input, otherwise it will use the \gls{wmi} protocol to execute a command and get the result
    \item In the case that an interactive shell is started, the tool will execute every command inputted by executing a remote command using the \gls{wmi} protocol, and thereafter return the result when the remote command has executed
\end{enumerate}

Now that we have a remote access to the host using either clear-text credentials or a hash, we can start the process of extracting memory form the \gls{lsass} process. As mentioned, this process will contain all credentials for currently active sessions.

\paragraph{Dumping \gls{lsass} memory}
Dumping memory of a process in Windows is a somewhat difficult tasks when you do not have a graphical interface. Using a graphical interface you can easily dump the memory of a process using Task Manager\cite{url:microsoft:dump-memory-task-manager}, but in the command line it is somewhat more difficult as no native ways exists. The solution is to use to tool Procdump from Microsoft's Sysinternal toolset.\\
With a WMIExec connection present we can do the following to dump and download the memory of a remote host
\begin{enumerate}
    \item Start WMIExec with a remote connection to the remote host
    \item Upload the Procdump tools to the remote host
    \item Run Procdump with the parameters \emph{-ma -accepteula lsass.exe debug.dmp} to save the \gls{lsass} process memory to debug.dmp
    \item Download the debug.dmp file to our own machine
    \item Delete Procdump and debug.dmp
    \item Close the connection
\end{enumerate}

\end{document}