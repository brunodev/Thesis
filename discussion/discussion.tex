\documentclass{article}
\usepackage{import}
\subimport{../}{preamble.tex}
\standalonetrue
\begin{document}

% sample content
\section{Discussion}
Section \ref{sec:initial-foothold} (\nameref{sec:initial-foothold}) and \ref{sec:credential-acquiring} (\nameref{sec:credential-acquiring}) analyzed and explained in detail how attacking Windows \gls{ad} Domain can be executed in theory and \ref{sec:implementation} (\nameref{sec:implementation}) described how the theory was applied in real life and implemented in an web application.
\\\\
To really understand exactly what happens and in what order, the following list will show the full attack chain done through the web app.

\begin{enumerate}
    \item The spoofers and services are started
    \item Host \textbf{client} tries to access \textbf{server1} using \gls{smb}, but \textbf{server1} is not registered in the DNS server.
    \item The \textbf{client} therefore sends both \gls{llmnr} and \gls{nbns} packets to the local subnet
    \item The \textbf{attacker} sends spoofed \gls{llmnr} and \gls{nbns} packets to \textbf{client} and successfully redirect traffic to itself
    \item The \textbf{client} authenticates using NTLM with the user \textbf{user1} to \textbf{attacker}, and the NetNTLMv2 hash is now caught be the implemented \gls{smb} server. The user is now visible in the web app.
    \item The NetNTLMv2 hash is dumped from the web app and bruteforced using hashcat
    \item The newly acquired user is added using the web app
    \item A target, \textbf{target1} that \textbf{user1} has access to is added to the target list using the web app
    \item The target \textbf{target1} is dumped using the web app and all cached credentials is added to the user list.
    \item Step 9 is then repeated for other targets that any of the users have access to.
    \item Domain admin credentials acquired on \textbf{targetX} and the stored \gls{json} files will be used to generate a timeline of actions done.
\end{enumerate}

If we look closer at the graphical interface of the web app as seen in appendix \ref{appendix:web-ui} and \ref{appendix:add-user-modal} it is very simple, and has the necessary features needed. The log will contain all actions done, and the user and target components will show current users and targets. Everything is automatically updated using SignalR and all actions done are synchronized both ways.
\\\\
In terms of pure functionality, the web app implements the full attack chain discovered in the theory sections. The spoofers and servers/services starts listening to packets on startup and the responses all work correctly such that NetNTLMv2 hashes are acquired. This was tested on multiple production networks as a part of the authors daily work, and it worked flawlessly.\footnote{The exact networks can not be disclosed due to \gls{nda}, but one of them were my own workplace}. Furthermore tests on a small virtual network consisting of a \gls{dc} and two standard windows machines were all successful in obtaining NetNTLMv2 hashes, cracking them manually and then dumping credentials using the "Dump" button as seen on appendix \ref{appendix:web-ui}.

\subimport{}{missing-features.tex}
\subimport{}{future.tex}
\subimport{}{reflection.tex}
\end{document}